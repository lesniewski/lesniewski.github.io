\documentclass[margin,line,11pt]{resume}
% vim:spell:

%\usepackage{lucida}

\oddsidemargin  -0.52in
\evensidemargin -0.52in
\textwidth       6.04in
\parskip 0.25\baselineskip plus 0.75\baselineskip
%\parsep 2pt plus 2pt
%\clubpenalty  4000
%\widowpenalty 4000


\usepackage{color}
\definecolor{darkblue}{rgb}{0,0,0.5}
\usepackage[colorlinks=true,citecolor=darkblue,linkcolor=darkblue,urlcolor=darkblue]{hyperref}

\usepackage[protrusion=true,expansion=true]{microtype}
%\usepackage{hyphenat}


\let\oldsection=\section
\renewcommand{\section}[1]{\oldsection{\small\sc #1}}


\usepackage{xspace}
\def\Whanau{Wh\=a\-nau\xspace}
\hyphenation{Wha-nau}



\begin{document}

\name{\large Christopher T Lesniewski-Laas}
\input{cv_updated.tex}
\address{Updated \updated}

\begin{resume}

\section{Contact Information}
MIT Computer Science and Artificial Intelligence Laboratory \hfill +1 617 253 0004      \\
32 Vassar Street 32-G996                                    \hfill \texttt{ctl@mit.edu} \\ 
Cambridge, MA 02139                                         \hfill \href{http://lesniewski.org}{lesniewski.org} \\


\section{Research Interests}
Distributed and decentralized computer systems, communication networks, social
networks, scaling, and security.


\section{Education}

\textbf{Massachusetts Institute of Technology}

Ph.D. candidate, Computer Science \hfill 2003--2010 (expected) \\
Dissertation: \textsl{A Secure and Decentralized Distributed Hash Table} \\
Advisor: Professor M. Frans Kaashoek

M.Eng., Electrical Engineering and Computer Science \hfill 2001--2003 \\
Thesis: \textsl{SSL Splitting and Barnraising: Cooperative Caching with Authenticity Guarantees}

S.B., Electrical Engineering and Computer Science \hfill 1997--2003 \\
S.B., Mathematics \hfill 1997--2001 \\
Minor: Physics


\section{Employment}
\textbf{MIT CSAIL},
Cambridge, MA
\hfill
2001--Present\\
Research Assistant,
Parallel and Distributed Operating Systems Group.

\textbf{University of Cambridge Computer Lab},
Cambridge, UK
\hfill
Summer 2004\\
Visiting scholar, Cambridge-MIT Institute, \textsl{``Next generation peer-to-peer networks''} project.

\textbf{Permabit},
Cambridge, MA
\hfill
Summer--Fall 2001\\
Designed and developed robust and scalable data storage system.

\textbf{Microsoft Research},
Redmond, WA
\hfill
Summer 2000\\
Designed cryptographic protocols for smart card based access control.

\textbf{SensAble Technologies},
Cambridge, MA
\hfill
Summer 1999\\
Developed hardware and software for a robotic haptic interface.

\textbf{MIT AI Lab, Mathematics and Computation},
Cambridge, MA
\hfill
Summer 1998\\
Developed software to simulate amorphous computers.



\section{Teaching}

\textbf{Recitation Instructor} (position usually filled by faculty) \hfill Spring 2005\\
\textbf{Head Teaching Assistant} \hfill Spring 2004\\
\textbf{Teaching Assistant} \hfill Spring 2003\\
MIT 6.033:
\textsl{Computer Systems Engineering}

\textbf{Lab Assistant} \hfill Fall 2000\\
MIT 6.001:
\textsl{Structure and Interpretation of Computer Programs}

\newpage


% Publications
\documentclass[margin,line,11pt]{resume}
% vim:spell:

%\usepackage{lucida}

\oddsidemargin  -0.52in
\evensidemargin -0.52in
\textwidth       6.04in
\parskip 0.25\baselineskip plus 0.75\baselineskip
%\parsep 2pt plus 2pt
%\clubpenalty  4000
%\widowpenalty 4000


\usepackage{color}
\definecolor{darkblue}{rgb}{0,0,0.5}
\usepackage[colorlinks=true,citecolor=darkblue,linkcolor=darkblue,urlcolor=darkblue]{hyperref}

\usepackage[protrusion=true,expansion=true]{microtype}
%\usepackage{hyphenat}


\let\oldsection=\section
\renewcommand{\section}[1]{\oldsection{\small\sc #1}}


\usepackage{xspace}
\def\Whanau{Wh\=a\-nau\xspace}
\hyphenation{Wha-nau}



\begin{document}

\name{\large Christopher T Lesniewski-Laas}
\input{cv_updated.tex}
\address{Updated \updated}

\begin{resume}

\section{Contact Information}
MIT Computer Science and Artificial Intelligence Laboratory \hfill +1 617 253 0004      \\
32 Vassar Street 32-G996                                    \hfill \texttt{ctl@mit.edu} \\ 
Cambridge, MA 02139                                         \hfill \href{http://lesniewski.org}{lesniewski.org} \\


\section{Research Interests}
Distributed and decentralized computer systems, communication networks, social
networks, scaling, and security.


\section{Education}

\textbf{Massachusetts Institute of Technology}

Ph.D. candidate, Computer Science \hfill 2003--2010 (expected) \\
Dissertation: \textsl{A Secure and Decentralized Distributed Hash Table} \\
Advisor: Professor M. Frans Kaashoek

M.Eng., Electrical Engineering and Computer Science \hfill 2001--2003 \\
Thesis: \textsl{SSL Splitting and Barnraising: Cooperative Caching with Authenticity Guarantees}

S.B., Electrical Engineering and Computer Science \hfill 1997--2003 \\
S.B., Mathematics \hfill 1997--2001 \\
Minor: Physics


\section{Employment}
\textbf{MIT CSAIL},
Cambridge, MA
\hfill
2001--Present\\
Research Assistant,
Parallel and Distributed Operating Systems Group.

\textbf{University of Cambridge Computer Lab},
Cambridge, UK
\hfill
Summer 2004\\
Visiting scholar, Cambridge-MIT Institute, \textsl{``Next generation peer-to-peer networks''} project.

\textbf{Permabit},
Cambridge, MA
\hfill
Summer--Fall 2001\\
Designed and developed robust and scalable data storage system.

\textbf{Microsoft Research},
Redmond, WA
\hfill
Summer 2000\\
Designed cryptographic protocols for smart card based access control.

\textbf{SensAble Technologies},
Cambridge, MA
\hfill
Summer 1999\\
Developed hardware and software for a robotic haptic interface.

\textbf{MIT AI Lab, Mathematics and Computation},
Cambridge, MA
\hfill
Summer 1998\\
Developed software to simulate amorphous computers.



\section{Teaching}

\textbf{Recitation Instructor} (position usually filled by faculty) \hfill Spring 2005\\
\textbf{Head Teaching Assistant} \hfill Spring 2004\\
\textbf{Teaching Assistant} \hfill Spring 2003\\
MIT 6.033:
\textsl{Computer Systems Engineering}

\textbf{Lab Assistant} \hfill Fall 2000\\
MIT 6.001:
\textsl{Structure and Interpretation of Computer Programs}

\newpage


% Publications
\documentclass[margin,line,11pt]{resume}
% vim:spell:

%\usepackage{lucida}

\oddsidemargin  -0.52in
\evensidemargin -0.52in
\textwidth       6.04in
\parskip 0.25\baselineskip plus 0.75\baselineskip
%\parsep 2pt plus 2pt
%\clubpenalty  4000
%\widowpenalty 4000


\usepackage{color}
\definecolor{darkblue}{rgb}{0,0,0.5}
\usepackage[colorlinks=true,citecolor=darkblue,linkcolor=darkblue,urlcolor=darkblue]{hyperref}

\usepackage[protrusion=true,expansion=true]{microtype}
%\usepackage{hyphenat}


\let\oldsection=\section
\renewcommand{\section}[1]{\oldsection{\small\sc #1}}


\usepackage{xspace}
\def\Whanau{Wh\=a\-nau\xspace}
\hyphenation{Wha-nau}



\begin{document}

\name{\large Christopher T Lesniewski-Laas}
\input{cv_updated.tex}
\address{Updated \updated}

\begin{resume}

\section{Contact Information}
MIT Computer Science and Artificial Intelligence Laboratory \hfill +1 617 253 0004      \\
32 Vassar Street 32-G996                                    \hfill \texttt{ctl@mit.edu} \\ 
Cambridge, MA 02139                                         \hfill \href{http://lesniewski.org}{lesniewski.org} \\


\section{Research Interests}
Distributed and decentralized computer systems, communication networks, social
networks, scaling, and security.


\section{Education}

\textbf{Massachusetts Institute of Technology}

Ph.D. candidate, Computer Science \hfill 2003--2010 (expected) \\
Dissertation: \textsl{A Secure and Decentralized Distributed Hash Table} \\
Advisor: Professor M. Frans Kaashoek

M.Eng., Electrical Engineering and Computer Science \hfill 2001--2003 \\
Thesis: \textsl{SSL Splitting and Barnraising: Cooperative Caching with Authenticity Guarantees}

S.B., Electrical Engineering and Computer Science \hfill 1997--2003 \\
S.B., Mathematics \hfill 1997--2001 \\
Minor: Physics


\section{Employment}
\textbf{MIT CSAIL},
Cambridge, MA
\hfill
2001--Present\\
Research Assistant,
Parallel and Distributed Operating Systems Group.

\textbf{University of Cambridge Computer Lab},
Cambridge, UK
\hfill
Summer 2004\\
Visiting scholar, Cambridge-MIT Institute, \textsl{``Next generation peer-to-peer networks''} project.

\textbf{Permabit},
Cambridge, MA
\hfill
Summer--Fall 2001\\
Designed and developed robust and scalable data storage system.

\textbf{Microsoft Research},
Redmond, WA
\hfill
Summer 2000\\
Designed cryptographic protocols for smart card based access control.

\textbf{SensAble Technologies},
Cambridge, MA
\hfill
Summer 1999\\
Developed hardware and software for a robotic haptic interface.

\textbf{MIT AI Lab, Mathematics and Computation},
Cambridge, MA
\hfill
Summer 1998\\
Developed software to simulate amorphous computers.



\section{Teaching}

\textbf{Recitation Instructor} (position usually filled by faculty) \hfill Spring 2005\\
\textbf{Head Teaching Assistant} \hfill Spring 2004\\
\textbf{Teaching Assistant} \hfill Spring 2003\\
MIT 6.033:
\textsl{Computer Systems Engineering}

\textbf{Lab Assistant} \hfill Fall 2000\\
MIT 6.001:
\textsl{Structure and Interpretation of Computer Programs}

\newpage


% Publications
\documentclass[margin,line,11pt]{resume}
% vim:spell:

%\usepackage{lucida}

\oddsidemargin  -0.52in
\evensidemargin -0.52in
\textwidth       6.04in
\parskip 0.25\baselineskip plus 0.75\baselineskip
%\parsep 2pt plus 2pt
%\clubpenalty  4000
%\widowpenalty 4000


\usepackage{color}
\definecolor{darkblue}{rgb}{0,0,0.5}
\usepackage[colorlinks=true,citecolor=darkblue,linkcolor=darkblue,urlcolor=darkblue]{hyperref}

\usepackage[protrusion=true,expansion=true]{microtype}
%\usepackage{hyphenat}


\let\oldsection=\section
\renewcommand{\section}[1]{\oldsection{\small\sc #1}}


\usepackage{xspace}
\def\Whanau{Wh\=a\-nau\xspace}
\hyphenation{Wha-nau}



\begin{document}

\name{\large Christopher T Lesniewski-Laas}
\input{cv_updated.tex}
\address{Updated \updated}

\begin{resume}

\section{Contact Information}
MIT Computer Science and Artificial Intelligence Laboratory \hfill +1 617 253 0004      \\
32 Vassar Street 32-G996                                    \hfill \texttt{ctl@mit.edu} \\ 
Cambridge, MA 02139                                         \hfill \href{http://lesniewski.org}{lesniewski.org} \\


\section{Research Interests}
Distributed and decentralized computer systems, communication networks, social
networks, scaling, and security.


\section{Education}

\textbf{Massachusetts Institute of Technology}

Ph.D. candidate, Computer Science \hfill 2003--2010 (expected) \\
Dissertation: \textsl{A Secure and Decentralized Distributed Hash Table} \\
Advisor: Professor M. Frans Kaashoek

M.Eng., Electrical Engineering and Computer Science \hfill 2001--2003 \\
Thesis: \textsl{SSL Splitting and Barnraising: Cooperative Caching with Authenticity Guarantees}

S.B., Electrical Engineering and Computer Science \hfill 1997--2003 \\
S.B., Mathematics \hfill 1997--2001 \\
Minor: Physics


\section{Employment}
\textbf{MIT CSAIL},
Cambridge, MA
\hfill
2001--Present\\
Research Assistant,
Parallel and Distributed Operating Systems Group.

\textbf{University of Cambridge Computer Lab},
Cambridge, UK
\hfill
Summer 2004\\
Visiting scholar, Cambridge-MIT Institute, \textsl{``Next generation peer-to-peer networks''} project.

\textbf{Permabit},
Cambridge, MA
\hfill
Summer--Fall 2001\\
Designed and developed robust and scalable data storage system.

\textbf{Microsoft Research},
Redmond, WA
\hfill
Summer 2000\\
Designed cryptographic protocols for smart card based access control.

\textbf{SensAble Technologies},
Cambridge, MA
\hfill
Summer 1999\\
Developed hardware and software for a robotic haptic interface.

\textbf{MIT AI Lab, Mathematics and Computation},
Cambridge, MA
\hfill
Summer 1998\\
Developed software to simulate amorphous computers.



\section{Teaching}

\textbf{Recitation Instructor} (position usually filled by faculty) \hfill Spring 2005\\
\textbf{Head Teaching Assistant} \hfill Spring 2004\\
\textbf{Teaching Assistant} \hfill Spring 2003\\
MIT 6.033:
\textsl{Computer Systems Engineering}

\textbf{Lab Assistant} \hfill Fall 2000\\
MIT 6.001:
\textsl{Structure and Interpretation of Computer Programs}

\newpage


% Publications
\input{cv.bbl}


\section{External Talks}

Yale University, New Haven, CT \hfill February 2010\\
\textsl{A Sybil-proof Distributed Hash Table.}

Microsoft Research, Redmond, WA \hfill December 2008\\
\textsl{Defending against Sybils using the social network.}

Nokia, Oulu, Finland \hfill June 2008\\
\textsl{Compact Internet routing.}

EuroSys Workshop on Social Network Systems, Glasgow, UK \hfill April 2008\\
\textsl{A Sybil-proof DHT using a social network.}

University of Cambridge Computer Laboratory, Cambridge, UK \hfill March 2008\\
\textsl{A Sybil-proof DHT using a social network.}

Nokia Research, Cambridge, MA \hfill January 2008\\
\textsl{Alpaca, a really flexible authentication framework.}

ACM Conference on Computer and Communications Security (CCS) \hfill October 2007\\
\textsl{Extensible proof-carrying authorization in Alpaca.}

IRIS Student Workshop \hfill November 2004\\
\textsl{Does overlay routing security require admission control?}

Johns Hopkins University, Baltimore, MD \hfill August 2004\\
\textsl{SSL Splitting.}

USENIX Security Symposium \hfill August 2003\\
\textsl{SSL Splitting.}

%Demos and posters?



\section{Software Artifacts}

\textbf{\Whanau}:
secure and scalable distributed hash table.
\hfill 2010 \\
\textbf{Eyo}:
device-transparent personal storage system.
\hfill 2009 \\
\textbf{Alpaca}:
extensible proof-carrying-authorization framework library.
\hfill 2007 \\
\textbf{UIA}:
naming and routing protocol suite for personal mobile devices.
\hfill 2006 \\
\textbf{Barnraising}:
distributed caching Web proxy using SSL Splitting.
\hfill 2003 \\
\textbf{SSL Splitting}:
drop-in replacement for OpenSSL library enabling untrusted caches.
\hfill 2003 \\

%AF, RSAGL, RTMP


\section{Professional Activities}

Program Committee, ACM Symposium on Applied Computing (Security Track),
2007--2010.

External reviews include:
SOSP 2003,2005,2007,
SIGCOMM 2003,
IPTPS 2003,
NDSS 2004,
J. Computer Networks (2004),
FAST 2005,
CCS 2006,
ISIT 2009,
SNS 2009,
Trans. Vehicular Tech (2009),
TISSEC (2010).


%\section{Awards}
%Kosciuszko Foundation Fellowship \hfill 2001\\
%Phi Beta Kappa \hfill 2001\\
%Microsoft I-Campus grant \hfill 2000

\section{Affiliations and Honors}
%\section{Affiliations}
ACM, USENIX, SIPB, Phi Beta Kappa, Kosciuszko Foundation Fellowship.



%\section{References}

%Prof. M. Frans Kaashoek\\
%MIT CSAIL\\
%32 Vassar St 32-G992\\
%Cambridge, MA 02139\\
%+1 617 253 7149\\
%\texttt{kaashoek@csail.mit.edu}

%Prof. Robert T. Morris\\
%MIT CSAIL\\
%32 Vassar St 32-G972\\
%Cambridge, MA 02139\\
%+1 617 253 5983\\
%\texttt{rtm@csail.mit.edu}

%Liskov
%Karger


\newpage

\section{Research Projects}

\textbf{\Whanau{} --- Sybil-proof Secure Distributed Hash Table}
\hfill 2008--Present\\
The topic of my dissertation, \Whanau is a secure
Distributed Hash Table (DHT): a structured overlay which can quickly look up
the node responsible for a given key.  DHTs have many applications, including
key-value databases, filesystems, caching, rendezvous services, and multicast
trees.  \Whanau uses the high connectivity of natural social networks
to bootstrap a highly robust overlay network.  Any attacker must infiltrate a
large fraction of the social network in order to cause any damage to the
system's availability.  In previous DHTs, an attacker can cause a massive
Denial of Service (DoS) simply by creating a large number of pseudonyms.
Previous defenses against this ``Sybil attack'', a problem identified in 2001,
required a centralized gatekeeper which is somehow able to distinguish the good
identities from evil pseudonyms.  For example, Amazon's Dynamo DHT operates
only within Amazon's data centers, and CoralCDN's DHT contains only PlanetLab
servers.  \Whanau eliminates this admission control function, enabling truly
decentralized and cooperative P2P DHT infrastructure to be built.

This work appeared at SocialNets 2008~\cite{sybil:socialnets08}
and will appear at NSDI 2010~\cite{whanau:nsdi10}.
An earlier paper in ESORICS 2005~\cite{sybil:esorics05} introduced the social
network model later used by \Whanau.

\textbf{UIA --- User Information Architecture}
\hfill 2004--2008\\
\textbf{Eyo --- Device Transparent Storage}
\hfill 2008--Present\\
UIA is a routing and naming layer designed to organize users' many
personal devices, such as laptops, phones, cameras, and media players, into a
coherent cluster.  Users introduce their devices to each other using a secure
physical rendezvous in which the devices exchange cryptographic keys;
thereafter, UIA's routing layer ensures that the devices can find and contact
each other whenever they are connected to the same network.  The user assigns
personal names to each device and UIA propagates records appropriately to
ensure that the namespace is consistent across all devices.  In addition, users
can assign names to other users, and can apply these names recursively to
navigate the social network.  For example, the name \texttt{phone.dad.bob}
would refer to \emph{Bob's father's telephone}.  Since no device or server is
designated as the ``master'' of a user's cluster, UIA's main challenge is
securely handling updates to the cluster's membership, including cases in which
some devices may be offline or acting maliciously.

Eyo, a continuation of the UIA project, tackles the problem of
providing a consistent view of a user's data objects (such as photos, music,
and email) across all of her devices.  We call  this property \emph{device
transparency}. As with UIA, the challenge is to provide a consistent view
despite varying device capabilities and network connectivity, and without
relying on a central master server.
A device transparent storage system must track object updates, forward changes
to running applications, handle concurrent updates, and proactively partition
and replicate data across heterogeneous devices.
Eyo addresses these requirements by separating objects' metadata from their
content, and distributing all metadata to all devices.

UIA appeared at IPTPS 2006~\cite{uia:iptps06}
and OSDI 2006~\cite{uia:osdi06}.
Eyo appeared at HotStorage 2009~\cite{eyo:hotstorage09}.


\textbf{Alpaca --- Extensible Proof-Carrying Authorization}
\hfill 2005--2008\\
Alpaca is a logic-based Proof-Carrying Authorization
framework.  It provides an API enabling network applications to state and prove
logical assertions such as ``the principal Alice says to delete the file X''
using cryptographic operations specified in the accompanying proof.  Since
verifiers don't care how the proof is structured, as long as it is valid, this
permits provers to use different cryptographic techniques (such as new hash
functions or data transport mechanisms) without breaking compatibility with
existing verifiers.  Alpaca's flexibility stands in contrast to cryptographic
protocols such as Kerberos and TLS, which can only be updated by installing new
software.  Crucially, Alpaca extensions do not need to be approved by any
central authority: any user can unilaterally deploy any extension as long as it
produces the correct type of proofs.  Extensions preserve security because they
do not expand users' privileges, they simply enable users to apply their
existing privileges in novel ways.

This work appeared at CCS 2007~\cite{alpaca:ccs07}.

\textbf{Barnraising and SSL Splitting --- Untrusted CDN}
\hfill 2002--2003\\
Barnraising is a P2P content distribution network (CDN) which,
like the later system CoralCDN, enables Web sites to delegate
some of their load to a distributed network of cooperating cache hosts.  Unlike
CoralCDN, Barnraising uses a novel technique called \emph{SSL Splitting} to
securely serve data using untrusted caches.  Because a malicious cache cannot
send clients bogus data, Barnraising can safely permit any Internet host to
contribute cache space; on the other hand, CoralCDN is limited to the
resources available from the centrally-controlled (and under-provisioned)
PlanetLab.

The SSL Splitting library is installed on a Web server as a drop-in replacement
for the popular OpenSSL library, enabling the server to communicate with the
untrusted Barnraising cache nodes.  SSL Splitting does not require any changes
to Web clients.

This work appeared at USENIX Security 2003~\cite{ssl-splitting:usenixsecurity03}
and in \textit{Computer Networks,\/} August 2005~\cite{ssl-splitting:compnet05}.

\end{resume}
\end{document}



\section{External Talks}

Yale University, New Haven, CT \hfill February 2010\\
\textsl{A Sybil-proof Distributed Hash Table.}

Microsoft Research, Redmond, WA \hfill December 2008\\
\textsl{Defending against Sybils using the social network.}

Nokia, Oulu, Finland \hfill June 2008\\
\textsl{Compact Internet routing.}

EuroSys Workshop on Social Network Systems, Glasgow, UK \hfill April 2008\\
\textsl{A Sybil-proof DHT using a social network.}

University of Cambridge Computer Laboratory, Cambridge, UK \hfill March 2008\\
\textsl{A Sybil-proof DHT using a social network.}

Nokia Research, Cambridge, MA \hfill January 2008\\
\textsl{Alpaca, a really flexible authentication framework.}

ACM Conference on Computer and Communications Security (CCS) \hfill October 2007\\
\textsl{Extensible proof-carrying authorization in Alpaca.}

IRIS Student Workshop \hfill November 2004\\
\textsl{Does overlay routing security require admission control?}

Johns Hopkins University, Baltimore, MD \hfill August 2004\\
\textsl{SSL Splitting.}

USENIX Security Symposium \hfill August 2003\\
\textsl{SSL Splitting.}

%Demos and posters?



\section{Software Artifacts}

\textbf{\Whanau}:
secure and scalable distributed hash table.
\hfill 2010 \\
\textbf{Eyo}:
device-transparent personal storage system.
\hfill 2009 \\
\textbf{Alpaca}:
extensible proof-carrying-authorization framework library.
\hfill 2007 \\
\textbf{UIA}:
naming and routing protocol suite for personal mobile devices.
\hfill 2006 \\
\textbf{Barnraising}:
distributed caching Web proxy using SSL Splitting.
\hfill 2003 \\
\textbf{SSL Splitting}:
drop-in replacement for OpenSSL library enabling untrusted caches.
\hfill 2003 \\

%AF, RSAGL, RTMP


\section{Professional Activities}

Program Committee, ACM Symposium on Applied Computing (Security Track),
2007--2010.

External reviews include:
SOSP 2003,2005,2007,
SIGCOMM 2003,
IPTPS 2003,
NDSS 2004,
J. Computer Networks (2004),
FAST 2005,
CCS 2006,
ISIT 2009,
SNS 2009,
Trans. Vehicular Tech (2009),
TISSEC (2010).


%\section{Awards}
%Kosciuszko Foundation Fellowship \hfill 2001\\
%Phi Beta Kappa \hfill 2001\\
%Microsoft I-Campus grant \hfill 2000

\section{Affiliations and Honors}
%\section{Affiliations}
ACM, USENIX, SIPB, Phi Beta Kappa, Kosciuszko Foundation Fellowship.



%\section{References}

%Prof. M. Frans Kaashoek\\
%MIT CSAIL\\
%32 Vassar St 32-G992\\
%Cambridge, MA 02139\\
%+1 617 253 7149\\
%\texttt{kaashoek@csail.mit.edu}

%Prof. Robert T. Morris\\
%MIT CSAIL\\
%32 Vassar St 32-G972\\
%Cambridge, MA 02139\\
%+1 617 253 5983\\
%\texttt{rtm@csail.mit.edu}

%Liskov
%Karger


\newpage

\section{Research Projects}

\textbf{\Whanau{} --- Sybil-proof Secure Distributed Hash Table}
\hfill 2008--Present\\
The topic of my dissertation, \Whanau is a secure
Distributed Hash Table (DHT): a structured overlay which can quickly look up
the node responsible for a given key.  DHTs have many applications, including
key-value databases, filesystems, caching, rendezvous services, and multicast
trees.  \Whanau uses the high connectivity of natural social networks
to bootstrap a highly robust overlay network.  Any attacker must infiltrate a
large fraction of the social network in order to cause any damage to the
system's availability.  In previous DHTs, an attacker can cause a massive
Denial of Service (DoS) simply by creating a large number of pseudonyms.
Previous defenses against this ``Sybil attack'', a problem identified in 2001,
required a centralized gatekeeper which is somehow able to distinguish the good
identities from evil pseudonyms.  For example, Amazon's Dynamo DHT operates
only within Amazon's data centers, and CoralCDN's DHT contains only PlanetLab
servers.  \Whanau eliminates this admission control function, enabling truly
decentralized and cooperative P2P DHT infrastructure to be built.

This work appeared at SocialNets 2008~\cite{sybil:socialnets08}
and will appear at NSDI 2010~\cite{whanau:nsdi10}.
An earlier paper in ESORICS 2005~\cite{sybil:esorics05} introduced the social
network model later used by \Whanau.

\textbf{UIA --- User Information Architecture}
\hfill 2004--2008\\
\textbf{Eyo --- Device Transparent Storage}
\hfill 2008--Present\\
UIA is a routing and naming layer designed to organize users' many
personal devices, such as laptops, phones, cameras, and media players, into a
coherent cluster.  Users introduce their devices to each other using a secure
physical rendezvous in which the devices exchange cryptographic keys;
thereafter, UIA's routing layer ensures that the devices can find and contact
each other whenever they are connected to the same network.  The user assigns
personal names to each device and UIA propagates records appropriately to
ensure that the namespace is consistent across all devices.  In addition, users
can assign names to other users, and can apply these names recursively to
navigate the social network.  For example, the name \texttt{phone.dad.bob}
would refer to \emph{Bob's father's telephone}.  Since no device or server is
designated as the ``master'' of a user's cluster, UIA's main challenge is
securely handling updates to the cluster's membership, including cases in which
some devices may be offline or acting maliciously.

Eyo, a continuation of the UIA project, tackles the problem of
providing a consistent view of a user's data objects (such as photos, music,
and email) across all of her devices.  We call  this property \emph{device
transparency}. As with UIA, the challenge is to provide a consistent view
despite varying device capabilities and network connectivity, and without
relying on a central master server.
A device transparent storage system must track object updates, forward changes
to running applications, handle concurrent updates, and proactively partition
and replicate data across heterogeneous devices.
Eyo addresses these requirements by separating objects' metadata from their
content, and distributing all metadata to all devices.

UIA appeared at IPTPS 2006~\cite{uia:iptps06}
and OSDI 2006~\cite{uia:osdi06}.
Eyo appeared at HotStorage 2009~\cite{eyo:hotstorage09}.


\textbf{Alpaca --- Extensible Proof-Carrying Authorization}
\hfill 2005--2008\\
Alpaca is a logic-based Proof-Carrying Authorization
framework.  It provides an API enabling network applications to state and prove
logical assertions such as ``the principal Alice says to delete the file X''
using cryptographic operations specified in the accompanying proof.  Since
verifiers don't care how the proof is structured, as long as it is valid, this
permits provers to use different cryptographic techniques (such as new hash
functions or data transport mechanisms) without breaking compatibility with
existing verifiers.  Alpaca's flexibility stands in contrast to cryptographic
protocols such as Kerberos and TLS, which can only be updated by installing new
software.  Crucially, Alpaca extensions do not need to be approved by any
central authority: any user can unilaterally deploy any extension as long as it
produces the correct type of proofs.  Extensions preserve security because they
do not expand users' privileges, they simply enable users to apply their
existing privileges in novel ways.

This work appeared at CCS 2007~\cite{alpaca:ccs07}.

\textbf{Barnraising and SSL Splitting --- Untrusted CDN}
\hfill 2002--2003\\
Barnraising is a P2P content distribution network (CDN) which,
like the later system CoralCDN, enables Web sites to delegate
some of their load to a distributed network of cooperating cache hosts.  Unlike
CoralCDN, Barnraising uses a novel technique called \emph{SSL Splitting} to
securely serve data using untrusted caches.  Because a malicious cache cannot
send clients bogus data, Barnraising can safely permit any Internet host to
contribute cache space; on the other hand, CoralCDN is limited to the
resources available from the centrally-controlled (and under-provisioned)
PlanetLab.

The SSL Splitting library is installed on a Web server as a drop-in replacement
for the popular OpenSSL library, enabling the server to communicate with the
untrusted Barnraising cache nodes.  SSL Splitting does not require any changes
to Web clients.

This work appeared at USENIX Security 2003~\cite{ssl-splitting:usenixsecurity03}
and in \textit{Computer Networks,\/} August 2005~\cite{ssl-splitting:compnet05}.

\end{resume}
\end{document}



\section{External Talks}

Yale University, New Haven, CT \hfill February 2010\\
\textsl{A Sybil-proof Distributed Hash Table.}

Microsoft Research, Redmond, WA \hfill December 2008\\
\textsl{Defending against Sybils using the social network.}

Nokia, Oulu, Finland \hfill June 2008\\
\textsl{Compact Internet routing.}

EuroSys Workshop on Social Network Systems, Glasgow, UK \hfill April 2008\\
\textsl{A Sybil-proof DHT using a social network.}

University of Cambridge Computer Laboratory, Cambridge, UK \hfill March 2008\\
\textsl{A Sybil-proof DHT using a social network.}

Nokia Research, Cambridge, MA \hfill January 2008\\
\textsl{Alpaca, a really flexible authentication framework.}

ACM Conference on Computer and Communications Security (CCS) \hfill October 2007\\
\textsl{Extensible proof-carrying authorization in Alpaca.}

IRIS Student Workshop \hfill November 2004\\
\textsl{Does overlay routing security require admission control?}

Johns Hopkins University, Baltimore, MD \hfill August 2004\\
\textsl{SSL Splitting.}

USENIX Security Symposium \hfill August 2003\\
\textsl{SSL Splitting.}

%Demos and posters?



\section{Software Artifacts}

\textbf{\Whanau}:
secure and scalable distributed hash table.
\hfill 2010 \\
\textbf{Eyo}:
device-transparent personal storage system.
\hfill 2009 \\
\textbf{Alpaca}:
extensible proof-carrying-authorization framework library.
\hfill 2007 \\
\textbf{UIA}:
naming and routing protocol suite for personal mobile devices.
\hfill 2006 \\
\textbf{Barnraising}:
distributed caching Web proxy using SSL Splitting.
\hfill 2003 \\
\textbf{SSL Splitting}:
drop-in replacement for OpenSSL library enabling untrusted caches.
\hfill 2003 \\

%AF, RSAGL, RTMP


\section{Professional Activities}

Program Committee, ACM Symposium on Applied Computing (Security Track),
2007--2010.

External reviews include:
SOSP 2003,2005,2007,
SIGCOMM 2003,
IPTPS 2003,
NDSS 2004,
J. Computer Networks (2004),
FAST 2005,
CCS 2006,
ISIT 2009,
SNS 2009,
Trans. Vehicular Tech (2009),
TISSEC (2010).


%\section{Awards}
%Kosciuszko Foundation Fellowship \hfill 2001\\
%Phi Beta Kappa \hfill 2001\\
%Microsoft I-Campus grant \hfill 2000

\section{Affiliations and Honors}
%\section{Affiliations}
ACM, USENIX, SIPB, Phi Beta Kappa, Kosciuszko Foundation Fellowship.



%\section{References}

%Prof. M. Frans Kaashoek\\
%MIT CSAIL\\
%32 Vassar St 32-G992\\
%Cambridge, MA 02139\\
%+1 617 253 7149\\
%\texttt{kaashoek@csail.mit.edu}

%Prof. Robert T. Morris\\
%MIT CSAIL\\
%32 Vassar St 32-G972\\
%Cambridge, MA 02139\\
%+1 617 253 5983\\
%\texttt{rtm@csail.mit.edu}

%Liskov
%Karger


\newpage

\section{Research Projects}

\textbf{\Whanau{} --- Sybil-proof Secure Distributed Hash Table}
\hfill 2008--Present\\
The topic of my dissertation, \Whanau is a secure
Distributed Hash Table (DHT): a structured overlay which can quickly look up
the node responsible for a given key.  DHTs have many applications, including
key-value databases, filesystems, caching, rendezvous services, and multicast
trees.  \Whanau uses the high connectivity of natural social networks
to bootstrap a highly robust overlay network.  Any attacker must infiltrate a
large fraction of the social network in order to cause any damage to the
system's availability.  In previous DHTs, an attacker can cause a massive
Denial of Service (DoS) simply by creating a large number of pseudonyms.
Previous defenses against this ``Sybil attack'', a problem identified in 2001,
required a centralized gatekeeper which is somehow able to distinguish the good
identities from evil pseudonyms.  For example, Amazon's Dynamo DHT operates
only within Amazon's data centers, and CoralCDN's DHT contains only PlanetLab
servers.  \Whanau eliminates this admission control function, enabling truly
decentralized and cooperative P2P DHT infrastructure to be built.

This work appeared at SocialNets 2008~\cite{sybil:socialnets08}
and will appear at NSDI 2010~\cite{whanau:nsdi10}.
An earlier paper in ESORICS 2005~\cite{sybil:esorics05} introduced the social
network model later used by \Whanau.

\textbf{UIA --- User Information Architecture}
\hfill 2004--2008\\
\textbf{Eyo --- Device Transparent Storage}
\hfill 2008--Present\\
UIA is a routing and naming layer designed to organize users' many
personal devices, such as laptops, phones, cameras, and media players, into a
coherent cluster.  Users introduce their devices to each other using a secure
physical rendezvous in which the devices exchange cryptographic keys;
thereafter, UIA's routing layer ensures that the devices can find and contact
each other whenever they are connected to the same network.  The user assigns
personal names to each device and UIA propagates records appropriately to
ensure that the namespace is consistent across all devices.  In addition, users
can assign names to other users, and can apply these names recursively to
navigate the social network.  For example, the name \texttt{phone.dad.bob}
would refer to \emph{Bob's father's telephone}.  Since no device or server is
designated as the ``master'' of a user's cluster, UIA's main challenge is
securely handling updates to the cluster's membership, including cases in which
some devices may be offline or acting maliciously.

Eyo, a continuation of the UIA project, tackles the problem of
providing a consistent view of a user's data objects (such as photos, music,
and email) across all of her devices.  We call  this property \emph{device
transparency}. As with UIA, the challenge is to provide a consistent view
despite varying device capabilities and network connectivity, and without
relying on a central master server.
A device transparent storage system must track object updates, forward changes
to running applications, handle concurrent updates, and proactively partition
and replicate data across heterogeneous devices.
Eyo addresses these requirements by separating objects' metadata from their
content, and distributing all metadata to all devices.

UIA appeared at IPTPS 2006~\cite{uia:iptps06}
and OSDI 2006~\cite{uia:osdi06}.
Eyo appeared at HotStorage 2009~\cite{eyo:hotstorage09}.


\textbf{Alpaca --- Extensible Proof-Carrying Authorization}
\hfill 2005--2008\\
Alpaca is a logic-based Proof-Carrying Authorization
framework.  It provides an API enabling network applications to state and prove
logical assertions such as ``the principal Alice says to delete the file X''
using cryptographic operations specified in the accompanying proof.  Since
verifiers don't care how the proof is structured, as long as it is valid, this
permits provers to use different cryptographic techniques (such as new hash
functions or data transport mechanisms) without breaking compatibility with
existing verifiers.  Alpaca's flexibility stands in contrast to cryptographic
protocols such as Kerberos and TLS, which can only be updated by installing new
software.  Crucially, Alpaca extensions do not need to be approved by any
central authority: any user can unilaterally deploy any extension as long as it
produces the correct type of proofs.  Extensions preserve security because they
do not expand users' privileges, they simply enable users to apply their
existing privileges in novel ways.

This work appeared at CCS 2007~\cite{alpaca:ccs07}.

\textbf{Barnraising and SSL Splitting --- Untrusted CDN}
\hfill 2002--2003\\
Barnraising is a P2P content distribution network (CDN) which,
like the later system CoralCDN, enables Web sites to delegate
some of their load to a distributed network of cooperating cache hosts.  Unlike
CoralCDN, Barnraising uses a novel technique called \emph{SSL Splitting} to
securely serve data using untrusted caches.  Because a malicious cache cannot
send clients bogus data, Barnraising can safely permit any Internet host to
contribute cache space; on the other hand, CoralCDN is limited to the
resources available from the centrally-controlled (and under-provisioned)
PlanetLab.

The SSL Splitting library is installed on a Web server as a drop-in replacement
for the popular OpenSSL library, enabling the server to communicate with the
untrusted Barnraising cache nodes.  SSL Splitting does not require any changes
to Web clients.

This work appeared at USENIX Security 2003~\cite{ssl-splitting:usenixsecurity03}
and in \textit{Computer Networks,\/} August 2005~\cite{ssl-splitting:compnet05}.

\end{resume}
\end{document}



\section{External Talks}

Yale University, New Haven, CT \hfill February 2010\\
\textsl{A Sybil-proof Distributed Hash Table.}

Microsoft Research, Redmond, WA \hfill December 2008\\
\textsl{Defending against Sybils using the social network.}

Nokia, Oulu, Finland \hfill June 2008\\
\textsl{Compact Internet routing.}

EuroSys Workshop on Social Network Systems, Glasgow, UK \hfill April 2008\\
\textsl{A Sybil-proof DHT using a social network.}

University of Cambridge Computer Laboratory, Cambridge, UK \hfill March 2008\\
\textsl{A Sybil-proof DHT using a social network.}

Nokia Research, Cambridge, MA \hfill January 2008\\
\textsl{Alpaca, a really flexible authentication framework.}

ACM Conference on Computer and Communications Security (CCS) \hfill October 2007\\
\textsl{Extensible proof-carrying authorization in Alpaca.}

IRIS Student Workshop \hfill November 2004\\
\textsl{Does overlay routing security require admission control?}

Johns Hopkins University, Baltimore, MD \hfill August 2004\\
\textsl{SSL Splitting.}

USENIX Security Symposium \hfill August 2003\\
\textsl{SSL Splitting.}

%Demos and posters?



\section{Software Artifacts}

\textbf{\Whanau}:
secure and scalable distributed hash table.
\hfill 2010 \\
\textbf{Eyo}:
device-transparent personal storage system.
\hfill 2009 \\
\textbf{Alpaca}:
extensible proof-carrying-authorization framework library.
\hfill 2007 \\
\textbf{UIA}:
naming and routing protocol suite for personal mobile devices.
\hfill 2006 \\
\textbf{Barnraising}:
distributed caching Web proxy using SSL Splitting.
\hfill 2003 \\
\textbf{SSL Splitting}:
drop-in replacement for OpenSSL library enabling untrusted caches.
\hfill 2003 \\

%AF, RSAGL, RTMP


\section{Professional Activities}

Program Committee, ACM Symposium on Applied Computing (Security Track),
2007--2010.

External reviews include:
SOSP 2003,2005,2007,
SIGCOMM 2003,
IPTPS 2003,
NDSS 2004,
J. Computer Networks (2004),
FAST 2005,
CCS 2006,
ISIT 2009,
SNS 2009,
Trans. Vehicular Tech (2009),
TISSEC (2010).


%\section{Awards}
%Kosciuszko Foundation Fellowship \hfill 2001\\
%Phi Beta Kappa \hfill 2001\\
%Microsoft I-Campus grant \hfill 2000

\section{Affiliations and Honors}
%\section{Affiliations}
ACM, USENIX, SIPB, Phi Beta Kappa, Kosciuszko Foundation Fellowship.



%\section{References}

%Prof. M. Frans Kaashoek\\
%MIT CSAIL\\
%32 Vassar St 32-G992\\
%Cambridge, MA 02139\\
%+1 617 253 7149\\
%\texttt{kaashoek@csail.mit.edu}

%Prof. Robert T. Morris\\
%MIT CSAIL\\
%32 Vassar St 32-G972\\
%Cambridge, MA 02139\\
%+1 617 253 5983\\
%\texttt{rtm@csail.mit.edu}

%Liskov
%Karger


\newpage

\section{Research Projects}

\textbf{\Whanau{} --- Sybil-proof Secure Distributed Hash Table}
\hfill 2008--Present\\
The topic of my dissertation, \Whanau is a secure
Distributed Hash Table (DHT): a structured overlay which can quickly look up
the node responsible for a given key.  DHTs have many applications, including
key-value databases, filesystems, caching, rendezvous services, and multicast
trees.  \Whanau uses the high connectivity of natural social networks
to bootstrap a highly robust overlay network.  Any attacker must infiltrate a
large fraction of the social network in order to cause any damage to the
system's availability.  In previous DHTs, an attacker can cause a massive
Denial of Service (DoS) simply by creating a large number of pseudonyms.
Previous defenses against this ``Sybil attack'', a problem identified in 2001,
required a centralized gatekeeper which is somehow able to distinguish the good
identities from evil pseudonyms.  For example, Amazon's Dynamo DHT operates
only within Amazon's data centers, and CoralCDN's DHT contains only PlanetLab
servers.  \Whanau eliminates this admission control function, enabling truly
decentralized and cooperative P2P DHT infrastructure to be built.

This work appeared at SocialNets 2008~\cite{sybil:socialnets08}
and will appear at NSDI 2010~\cite{whanau:nsdi10}.
An earlier paper in ESORICS 2005~\cite{sybil:esorics05} introduced the social
network model later used by \Whanau.

\textbf{UIA --- User Information Architecture}
\hfill 2004--2008\\
\textbf{Eyo --- Device Transparent Storage}
\hfill 2008--Present\\
UIA is a routing and naming layer designed to organize users' many
personal devices, such as laptops, phones, cameras, and media players, into a
coherent cluster.  Users introduce their devices to each other using a secure
physical rendezvous in which the devices exchange cryptographic keys;
thereafter, UIA's routing layer ensures that the devices can find and contact
each other whenever they are connected to the same network.  The user assigns
personal names to each device and UIA propagates records appropriately to
ensure that the namespace is consistent across all devices.  In addition, users
can assign names to other users, and can apply these names recursively to
navigate the social network.  For example, the name \texttt{phone.dad.bob}
would refer to \emph{Bob's father's telephone}.  Since no device or server is
designated as the ``master'' of a user's cluster, UIA's main challenge is
securely handling updates to the cluster's membership, including cases in which
some devices may be offline or acting maliciously.

Eyo, a continuation of the UIA project, tackles the problem of
providing a consistent view of a user's data objects (such as photos, music,
and email) across all of her devices.  We call  this property \emph{device
transparency}. As with UIA, the challenge is to provide a consistent view
despite varying device capabilities and network connectivity, and without
relying on a central master server.
A device transparent storage system must track object updates, forward changes
to running applications, handle concurrent updates, and proactively partition
and replicate data across heterogeneous devices.
Eyo addresses these requirements by separating objects' metadata from their
content, and distributing all metadata to all devices.

UIA appeared at IPTPS 2006~\cite{uia:iptps06}
and OSDI 2006~\cite{uia:osdi06}.
Eyo appeared at HotStorage 2009~\cite{eyo:hotstorage09}.


\textbf{Alpaca --- Extensible Proof-Carrying Authorization}
\hfill 2005--2008\\
Alpaca is a logic-based Proof-Carrying Authorization
framework.  It provides an API enabling network applications to state and prove
logical assertions such as ``the principal Alice says to delete the file X''
using cryptographic operations specified in the accompanying proof.  Since
verifiers don't care how the proof is structured, as long as it is valid, this
permits provers to use different cryptographic techniques (such as new hash
functions or data transport mechanisms) without breaking compatibility with
existing verifiers.  Alpaca's flexibility stands in contrast to cryptographic
protocols such as Kerberos and TLS, which can only be updated by installing new
software.  Crucially, Alpaca extensions do not need to be approved by any
central authority: any user can unilaterally deploy any extension as long as it
produces the correct type of proofs.  Extensions preserve security because they
do not expand users' privileges, they simply enable users to apply their
existing privileges in novel ways.

This work appeared at CCS 2007~\cite{alpaca:ccs07}.

\textbf{Barnraising and SSL Splitting --- Untrusted CDN}
\hfill 2002--2003\\
Barnraising is a P2P content distribution network (CDN) which,
like the later system CoralCDN, enables Web sites to delegate
some of their load to a distributed network of cooperating cache hosts.  Unlike
CoralCDN, Barnraising uses a novel technique called \emph{SSL Splitting} to
securely serve data using untrusted caches.  Because a malicious cache cannot
send clients bogus data, Barnraising can safely permit any Internet host to
contribute cache space; on the other hand, CoralCDN is limited to the
resources available from the centrally-controlled (and under-provisioned)
PlanetLab.

The SSL Splitting library is installed on a Web server as a drop-in replacement
for the popular OpenSSL library, enabling the server to communicate with the
untrusted Barnraising cache nodes.  SSL Splitting does not require any changes
to Web clients.

This work appeared at USENIX Security 2003~\cite{ssl-splitting:usenixsecurity03}
and in \textit{Computer Networks,\/} August 2005~\cite{ssl-splitting:compnet05}.

\end{resume}
\end{document}
